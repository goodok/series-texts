\documentclass[a4paper]{article}
\usepackage[utf8]{inputenc}
\usepackage[english]{babel}
\renewcommand{\familydefault}{\sfdefault}
\title{Series}
\author{}
\date{\today}

\pagestyle{headings}

\usepackage{amsfonts}
\usepackage{amssymb}
\usepackage{amsthm}
\usepackage{amscd}



\begin{document}

\tableofcontents

\section{Intro}
In order to clarify the discussions about Taylor series and asymptotic expansions, it is necessary to define the relevant concepts as precisely as possible.

Sympy makes a sharp distinction between expressions and functions. In the following, whenever something is defined for a function \(f\), there's an equivalent definition applying to the expression \(f(x)\).

\section{Definitions}

   \subsection{Series}
   
   In this section we attract attention to the structure of finite or infinite series.
    
        \begin{enumerate}
           \item The term \textbf{formal} which is used below for series usually means that "Historically, mathematicians such as Leonhard Euler operated liberally with infinite series, even if they were not
convergent. When calculus was put on a sound and correct foundation in the nineteenth century, rigorous proofs of
the convergence of series were always required. However, the formal operation with non-convergent series has been
retained in rings of formal power series which are studied in abstract algebra."

In other words, operations with formal series depend only upon the coefficients, the basis, not upon $X$ itself, so $X$ is a formal parameter. (They can be depend of $X$ nature: complex, real and so on.)

            \item A \textbf{Series} is the sum of the terms of a sequence. Finite sequences and series have defined first and last terms,
whereas infinite sequences and series continue indefinitely. The terms of the series are often produced according to a certain rule, such as by a formula, or by an algorithm. Remark about context:
When talking about series, one can refer either to the sequence ${S_N}$ of the partial sums, or to the result - the sum of the series.
            
            \item A \textbf{formal power series} is an object of the form $S = \sum_{n=0}^{+\infty} a_n X^n$ where $X$ is a formal parameter.
            
            The set of formal power series over a field $K$ is noted $K[[X]]$ and has a ring structure.
            
            \item  A \textbf{generalized formal power series}, or \textbf{formal Laurent series}, in an object $S$ such that $X^n S$ is a formal power series for some integer $n$. The set of formal Laurent series over a field $K$ is noted $K((X))$ and has a field structure.
            
            \item A classical \textbf{Laurent series} is an object of the form $\sum_{n=+\infty}^{+\infty} a_n X^n$.
            
        \end{enumerate}
        
   \subsection{Asymptotic expansion}
   
   In this section we attract attention to the asymptotic expansion and related things.
   
        \begin{enumerate}
            \item f is \textbf{dominated} by g at $x_0$, written $f = O_{x_0}(g)$, if $f/g$ is bounded at $x_0$. Similarly, $f(x) = O_{x \rightarrow x_0}(g(x))$ if $f = O_{x_0}(g)$, i.e. if $f(x)/g(x)$ is bounded for $x$ going to $x_0$. Note that $\cdot = O_\cdot(\cdot)$ is a purely notational device here, not an equality.
            
            \item The set of functions dominated by a function $g$ at $x_0$, $O_{x_0}(g)$ is defined by $f \in O_{x_0}(g)$ if $f = O_{x_0}(g)$. $h + O(g)$ is a set of functions defined in the usual way: $h + O(g) = \{h+\alpha ; \alpha \in O(g)\}$. It is called an \textbf{asymptotic expansion} of $f$ if $f \in h + O(g)$.
            
            \item In mathematics an \textbf{asymptotic expansion}, \textbf{asymptotic series} or \textbf{Poincaré expansion} (after Henri Poincaré) is a
formal series (infinit) of functions which has the property that truncating the series after a finite number of terms provides an
approximation to a given function as the argument of the function tends towards a particular, often infinite, point.

            \item \textbf{Asymptotic series}, otherwise \textbf{asymptotic expansions}, are infinite series whose partial sums become good
approximations in the limit of some point of the domain. In general they do not converge.

        \end{enumerate}
        
        
    \subsection{Generating function}
    
    This section is not really related with expansion but related with series. Roughly speaking, it is inversion of asymptotic series goals: while we have infinite number of coefficients of series then we obtain some function which is generated by them.
    
    "However such interpretation is not required to be possible, because formal power series are not required to give a
convergent series when a nonzero numeric value is substituted for x. Also, not all expressions that are meaningful as
functions of x are meaningful as expressions designating formal power series; negative and fractional powers of x are
examples of this. 
Generating functions are not functions in the formal sense of a mapping from a domain to a codomain; the name is
merely traditional, and they are sometimes more correctly called generating series."

    
    \begin{enumerate}
    
        \item The **ordinary generating function** of a sequence $a_n$ is $ G(a_n;x) = \sum_{n=0}^{+\infty} a_n x^n$.
        
        \item The **exponential generating function** of a sequence $a_n$ is $ EG(a_n;x) = \sum_{n=0}^{+\infty} a_n \frac{x^n}{n!}$.
    
    \end{enumerate}

\subsection{Usage and applications}

        
        \subsubsection{Series}
             \begin{enumerate}
                 \item One can use formal power series to prove several relations familiar from analysis in a purely algebraic setting.
                 \item In some cases it is easier to work with series, while processing with original function is difficult or impossible symbolically.
                 \item Some ODE can be solved by consideration of the desired result as formal series. Some recurrence formula for series coefficients can be obtained.
             \end{enumerate}

        \subsubsection{Asymptotic expansion}
             \begin{enumerate}
                \item Asymptotic expansions through sereies is used for approximation of functions. In some cases it is easier to work with truncated series, while processing with original function is difficult or impossible.
                \item Some ODE, Integral equations can be solved by consideration of the desired result as asymptotic expansions.
             \end{enumerate}

         \subsubsection{Generating function}
             \begin{enumerate}
                \item Find a closed formula for a sequence given in a recurrence relation.
                
                \item Find recurrence relations for sequences — the form of a generating function may suggest a recurrence formula.
                
                \item Define function through sequence.
                
                \item Discrete representation of function.
             \end{enumerate}
        
        \subsubsection{Common tasks}
             \begin{enumerate}
                \item Find a closed formula for a sequence given in a recurrence relation.
                
                \item Find recurrence relations for sequences — the form of a generating function may suggest a recurrence formula.
                
                \item Define function through sequence.
                
                \item Discrete representation of function.
             \end{enumerate}

    
\subsection{Sympy definitions}

$\sum_{k=a}^b a(n)$

\section{Current situation}

According to its docstring, ``f(x).series(x, x0, n)`` is supposed to return the (n-1)th order generalized Taylor expansion of $f(x)$ for $x \rightarrow x_0$.
Actually, it works only for $x_0 = 0$ and when f doesn't have a generalized Taylor expansion, it returns some arbitrarily chosen asymptotic expansion of $f(x)$.

But the present method "series" wich returns variose kinds of series is convinient and used for the task of limits processing: limits use nessesery amout of first terms of series whatever it be.
The work with processing of many various cases of series and limits was executed recently, also many tests have been collected and passed for series and limits.


Also there are an object Sum is defined, which represent unevaluated summation $\sum_{k=a}^b a(n) $.


Also the implementation of some series methods for solving IDEs is processing now in Saptarshi's branch (https://github.com/saptman/sympy/tree/dev_ide
).

\subsection{problems and remarks which we encounter}
    \begin{enumerate}
        \item representation of Derivative of function at some (no zero) point.
        \item Not effective algorithm in some cases: now is used that: (cos(x)*(sin(x)).series() = sin(x).series() * cos(x).series(), lseries.next() calculate the nseries(n)  every time  (f.e. fifth next() calculate nseries(5) and after this yield fifth term)
    \end{enumerate}

\section{Open questions}
    \begin{enumerate}
        multivariable extension
        complex number
    \end{enumerate}
\begin{itemize}
	\item 
\end{itemize}\end{document}
